% Karlin-Altschul Equation
%
% Basic introduction to the Karlin-Altschul equation.

\subsection{The Karlin-Altschul Equation}
\begin{frame}
  \frametitle{Definition}
  \begin{itemize}
    \item The Karlin-Altschul equation
    \begin{equation*}
      E = k m n e^{-\lambda S}
    \end{equation*}
    \item Symbols:
    \begin{itemize}
      \item $k$: minor constant, adjusts for correlation between alignments
      \item $m$: number of letters in query sequence
      \item $n$: number of letters in the database
      \item $\lambda$: scoring matrix scaling factor
      \item $S$: raw alignment score
    \end{itemize}
  \end{itemize}
\end{frame} 

\begin{frame}
  \frametitle{Interpretation}
  \begin{itemize}
    \item The Karlin-Altschul equation
    \begin{equation*}
      E = k m n e^{-\lambda S}
    \end{equation*}
    \item $E$ is the number of alignments of a similar score expected by chance when querying a database of the same size and letter frequency, where the letters in that database are randomly-ordered
    \item Small changes in score $S$ can produce large changes in $E$
    \item BUT biological sequence databases are not random!
  \end{itemize}
\end{frame} 