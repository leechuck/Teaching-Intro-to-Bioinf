% Assessing Prediction Methods
%
% Introduction to assessing and evaluating gene prediction

\subsection{Assessing Prediction Methods}
\begin{frame}
  \frametitle{Using a ``Gold Standard''}
  A general approach for \emph{all} predictive methods
  \begin{itemize}
    \item Define a known, ``correct'' set of true/false, positive/negative etc. examples - the ``gold standard''
    \item Evaluate your predictive method against that set for
    \begin{itemize}
      \item sensitivity, specificity, accuracy, precision, etc.
    \end{itemize}
  \end{itemize}
  Many methods available, coverage beyond the scope of this introduction
\end{frame}

\begin{frame}
  \frametitle{Contingency Tables}
  \begin{center}
     \begin{tabular}{cc|c|c|}
	  \cline{3-4}
		& & \multicolumn{2}{|c|}{Condition (Gold standard)}\\
	  \cline{3-4}
		& & True & False \\
	  \hline
	  \multicolumn{1}{ |c| }{\multirow{2}{*}{Test outcome}}& 
	  %\multicolumn{1}{ |c| }{Positive} & True Positive \cellcolor{green} & 
	  %  False Positive\cellcolor{red}\\
	  \multicolumn{1}{ |c| }{Positive} & True Positive  & 
	    False Positive\\
	  \cline{2-4}
	  \multicolumn{1}{ |c| }{} & \multicolumn{1}{ |c| }{Negative} & 
	    %False Negative\cellcolor{red} & True Negative \cellcolor{green}\\
	    False Negative & True Negative \\
	  \hline
    \end{tabular}
  \end{center}
  Sensitivity = TPR = $TP/(TP + FN)$ \\
  Specificity = TNR = $TN/(FP + TN)$ \\
  FPR = $1-\text{Specificity} = FP/(FP + TN)$ \\
  If you don't have this information, you can't interpret predictive results properly.
\end{frame}

\begin{frame}
  \frametitle{Why Performance Metrics Matter}
  \begin{itemize}
    \item<1-> You go for a checkup, and are tested for disease $X$
    \item<1-> The test has $\text{sensitivity}=0.95$ (predicts disease where there is disease)
    \item<1-> The test has $\text{FPR}=0.01$ (predicts disease where there is no disease)
    \item<2-> Your test is \emph{positive}
    \item<2-> What is the probability that you have disease $X$?
    \begin{itemize}
      \item 0.01, 0.05, 0.50, 0.95, 0.99?
    \end{itemize}
  \end{itemize} 
\end{frame}

\begin{frame}
  \frametitle{Why Performance Metrics Matter}
  \begin{itemize}
    \item<1-> What is the probability that you have disease $X$?
    \item<1-> Unless you know the \emph{baseline occurrence} of disease $X$, you cannot know.
    \item<2-> Baseline occurrence: $f_X$
    \begin{itemize}
      \item $f_X = 0.01 \implies P(\text{disease}|\text{+ve}) = 0.490 \approx 0.5$
      \item $f_X = 0.8 \implies P(\text{disease}|\text{+ve}) = 0.997 \approx 1.0$         
    \end{itemize}
  \end{itemize} 
\end{frame}

\begin{frame}
  \frametitle{Why Performance Metrics Matter}
  \begin{itemize}
    \item<1-> Imagine a predictor for protein functional class
    \item<1-> Predictor has has $\text{sensitivity}=0.95$, $\text{FPR}=0.01$
    \item<1-> You run the predictor on 20,000 proteins in an organism
    \item<2-> We estimate $\approx$ 200 members in protein complement, so $f_X=0.01$
    \begin{itemize}
      \item $f_X = 0.01 \implies P(\text{class}|\text{+ve}) = 0.490 \approx 0.5$
    \end{itemize}
  \end{itemize} 
\end{frame}

\begin{frame}
  \frametitle{Bayes' Theorem}
  \begin{itemize}
    \item May seem counter-intuitive: 95\% sensitivity, 99\% specificity $\implies$ 50\% chance of any prediction being incorrect
    \item Probability given by Bayes' Theorem
    \begin{itemize}
      \item $P(X|+) =  \frac{P(+|X) P(X)}{P(+|X) P(X) + P(+|\bar{X}) P(\bar{X})}$
    \end{itemize}
    \item This step commonly overlooked in the literature (confirmation bias?)
    \item e.g. in paper describing novel TTSS predictor: \\
      ``The surprisingly high number of (false) positives in genomes without TTSS exceeds the expected false positive rate"
  \end{itemize} 
\end{frame}

\begin{frame}
  \frametitle{Interpreting Performance Metrics}
  \begin{itemize}
    \item<1-> Use Bayes' Theorem!
    \item<1-> Predictions identify groups, not individual members of the group. e.g.
    \begin{itemize}
      \item Test for airport smugglers has $P(\text{smuggler}|+) = 0.9$
      \item Test gives 100 positives
    \end{itemize}
    \item<1-> Which specific individuals are truly smugglers?
    \item<2-> The test \emph{does not} allow you to determine this - you need more evidence for each individual
    \item<2->  Same principle applies to all other tests, (including protein functional class prediction) - you should not `cherry-pick' for publication without other evidence
  \end{itemize} 
\end{frame}

\begin{frame}
  \frametitle{``Gold Standard'' results}
  \begin{itemize}
    \item Tested \texttt{glimmer} and \texttt{prodigal} on two "gold standards"
    \begin{itemize}
      \item Manually annotated ($>$3 expert person years) close relative
      \item Community-annotated close relative
    \end{itemize}
    \item Both methods trained directly on the annotated genes in each organism!
  \end{itemize} 
\end{frame}

\begin{frame}
  \frametitle{``Gold Standard'' results}
  Manually annotated: 4550 CDS
  \begin{center}
    \begin{tabular}{r|l|l}
	  genecaller & \texttt{glimmer} & \texttt{prodigal}  \\
	  \hline
	  predicted & 4752    & 4287  \\
	  missed & 284 (6\%)   & 407 (9\%)  \\
	  \hline
	  \emph{Exact Prediction} & & \\
  	  sensitivity   & 62\%   & 71\%  \\
  	  FDR   & 41\%   & 25\%  \\  
	  PPV   & 59\% & 75\%  \\  
	  \hline
	  \emph{Correct ORF} & & \\
  	  sensitivity   & 94\%   & 91\% \\
  	  FDR   & 10\%  & 3\% \\  
	  PPV   & 90\% & 97\%  \\  
    \end{tabular}
  \end{center}     
\end{frame}

\begin{frame}
  \frametitle{``Gold Standard'' results}
  Community annotated: 4475 CDS
  \begin{center}
	\begin{tabular}{r|l|l}
	  genecaller & \texttt{glimmer} & \texttt{prodigal}  \\
	  \hline
	  predicted & 4679    & 4467  \\
	  missed & 112 (3\%)   & 156 (3\%)  \\
	  \hline
	  \emph{Exact Prediction} & & \\
  	  sensitivity   & 62\%   & 86\%  \\
  	  FDR   & 31\%   & 14\%  \\  
	  PPV   & 69\% & 86\%  \\  
	  \hline
	  \emph{Correct ORF} & & \\
  	  sensitivity   & 97\%   & 97\% \\
  	  FDR   & 7\%  & 3\% \\  
	  PPV   & 93\% & 97\%  \\  
	\end{tabular}
  \end{center}     
\end{frame}

\begin{frame}
   \frametitle{Gene/CDS Prediction}   
   \framesubtitle{Lessons learned}   
   \begin{itemize}
     \item Many alternative methods, all perform differently
     \item To assess/choose methods, performance metrics are required
     \item Even on (relatively simple) prokaryotes, current best methods imperfect
     \item Manual assessment and intervention is essential, and usually the most demanding part of the process
   \end{itemize}
\end{frame}